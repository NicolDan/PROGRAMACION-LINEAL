\documentclass{article}
\usepackage[utf8]{inputenc}
\usepackage{amsmath}
\usepackage[spanish]{babel}
\title{Apuntes de programación lineal}
\author{Daniela}

\setlength{\parindent}{0cm}
\begin{document}
\maketitle
\tableofcontents
\section{Forma estandar}
\label{sec:forma-estandar}


La forma estándar de un problema de programación líneal es:


Dados una matriz $A$ y vectores $b,c$, máximizar $c^Tx$ sujet a $Ax\leq b$


\section{Forma simplex}
\label{sec:forma-simplex}


Para escribir dicho problema en forma simplex se introduce una variable de holgura por cada una de las restricciones, para convertirlas en igualdades.
Entonces la forma simplex es:\\\\Dados una matriz $B$ y vectores $b,c$, máximizar $c^Tx$ sujeto a $Bx=b$ \\\\


\begin{tabular}{|c|c|c|}
  \hline
  & A & B \\
  \hline
  Máquina 1 & 1 & 2 \\
  \hline
  Máquina 2 & 1 & 1 \\
  \hline
    
\end{tabular}
\bigskip

\begin{equation}
  \label{eq:1}
  A=
  \begin{pmatrix}
    0 & 1 & 2 \\
    3 & -1 & 5
  \end{pmatrix}
  \begin{pmatrix}
    2 & 6 & 9\\
    4 & 3 & 0
   \end{pmatrix}
 \end{equation}
\end{document}

