\documentclass{article}

\usepackage[spanish]{babel}
\usepackage{amsmath}
\usepackage[utf8]{inputenc}
\title{Método Simplex}
\author{Daniela}

\begin{document}
\maketitle
\section{Introducción}
\label{sec:introduccion}

El método simplex es un algoritmo para resolver problemas de programación lineal. Fue inventado por George Dantzin en el año 1947

\section{Ejemplo}
Resuelve el problema por el método simplex.
\begin{equation*}
  \begin{aligned}
    \text{Maximizar} \quad & 2x_1+x_2\\
    \text{sujeto a} \quad &
    \begin{aligned}
      x_1+x_2 &\geq 1\\
      3x_1+4x_2 &\leq 12\\
      x_1-x_2 &\leq 2\\
      -2x_1+x_2 &\leq 2\\
      x_1,x_2 &\geq 0
     
    \end{aligned}
  \end{aligned}

\end{equation*}

Primeramente escribimos el problema en su forma estandar, para ellos multiplicamos la primera desigualdad por -1.\\

El problema es equivalente a:\\

\begin{equation*}
  \begin{aligned}
    \text{Maximizar} \quad & 2x_1+x_2\\
    \text{sujeto a} \quad &
    \begin{aligned}
      -x_1-x_2 &\leq -1\\
      3x_1+4x_2 &\leq 12\\
      x_1-x_2 &\leq 2\\
      -2x_1+x_2 &\leq 2\\
      x_1,x_2 &\geq 0
     
    \end{aligned}
  \end{aligned}

\end{equation*}
Una vez que hemos escrito el problema en forma estandar, procedemos a escribirlo en forma simplex, para ello agregamos una variable de houlgura por cada una de las desigualdades.
Sea $x_3, x_4, x_5, x_6 \geq 0$. El problema se transforma a:
\begin{equation*}
  \begin{aligned}
    \text{Maximizar} \quad & 2x_1+x_2\\
    \text{sujeto a} \quad &
    \begin{aligned}
      -x_1-x_2 + x_3 = -1\\
      3x_1+4x_2 +x_4= 12\\
      x_1-x_2 + x_5 =2\\
      -2x_1+x_2 + x_6=  2\\
      x_1,x_2,x_3, x_4, x_5, x_ &\geq 0
     
    \end{aligned}
  \end{aligned}

\end{equation*}

A conrinuación obtenemos un \emph{tablero simplex}, despejando las variables de holgura.
\begin{equation*}
  \begin{aligned}
      x_3& =-1+x_1+x_2\\
      x_4&= 12-3x_1-4x_2\\
      x_5&=2-x_1+x_2\\
      x_6&= 2 +2x_1-x_2\\
      \hline
      z&= \phantom{-1} 2x_1+x_2
     
    \end{aligned}
  \end{aligned}

\end{equation*}


\end{document}

